
\section{conceptos previos}

\subsection{Variable Regionalizada}

Es medida en el espacio de forma que presente una estructura de correlación, o de forma formal es un proceso estocástico con dominio contenido en un espacio euclidiano d-dimensional $\mathbb{R}^d, Z(\chi): \chi \in D \subset \mathbb{R}^d$, por ejemplo, si $d=2, z(\chi)$ puede asociarse a una variable medida en un punto $\chi$ en el plano

\subsection{Estacionariedad}

La variable regionalizada es estacionaria si su función de distribución conjunta es invariante respecto a cualquier translación del vector $h$. El concepto de estacionariedad es muy útil en la modelación de series temporales, y en ese contexto es fácil la identificación, puesto que sólo hay una dirección de variación (el tiempo) pero en el campo espacial existen múltiples direcciones y por lo tanto se debe asumir que en todas el fenómeno es estacionario. 

Cuando la esperanza de la variable no es la misma en todas las direcciones, dicho en otras palabras, una variable regionalizada será no estacionaria si su esperanza matemática no es constante o cuando la covarianza o correlación dependan del sentido en que se determinan, no habrá estacionariedad.


\subsection{Isotrópia y Anisotropía}

Si la correlación entre los datos no depende de la dirección en la que esta se calcule se dice que el fenómeno es isotrópico, en caso contrario se hablará de anisotropía
