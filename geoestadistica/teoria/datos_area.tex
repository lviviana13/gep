Cuando se trabajan con datos de corte transversal puede existir los siguientes problemas \cite{torres_ant}: 

\begin{itemize}
    \item Heterogeneidad espacial. Cuando, para explicar un fenómeno concreto,
     se utilizan datos procedentes de unidades espaciales muy diferentes entre sí ...           
     Este hecho puede conllevar la aparición de problemas de heteroscedasticidad y/o 
     inestabilidad estructural.

     \item Autocorrelación o dependencia espacial. Cuando existe una relación funcional 
     entre lo que ocurre en un punto determinado del espacio y lo que ocurre en otro lugar. 
     \begin{itemize}
        \item Autocorrelación Espacial Positiva: Cuando regiones “vecinas” tienden 
        a mostrar valores similares de una variable ... cuando valores elevados 
        o bajos de una variable tienden a estar agrupados en el espacio. 
        Un ejemplo sería el llamado efecto contagio o desbordamiento.
        \item Autocorrelación Espacial Negativa : Cuando localizaciones “vecinas” tienden a mostrar valores de una variable muy diferentes entre sí (casillas blancas y negras de un tablero de ajedrez). Un ejemplo sería un esquema Centro-Periferia.
        \item Distribución aleatoria: Ausencia de autocorrelación espacial.
     \end{itemize}

\end{itemize}

Existn dos posbile razones: La primera son errores de Medida, es decir, escasa correspondencia
 entre la extensión espacial del fenómeno bajo estudio y las unidades espaciales de observación.  
 AUTOCORRELACIÓN ESPÚRIA y la segunda es la interacción espacial entre unidades espaciales, 
 externalidades espaciales (spillovers): Efectos desbordamiento de las infraestructuras o 
 difusión tecnológica entre economías.
     
Como la autocorrelación espacial puede ser expresada por la covarianza  
pero esto requiere que la relación sea unidireccional, pero en el espacio la cosa es 
multidireccional entonces se requiere una notación espcifica que seria: 
OPERADOR RETARDO ESPACIAL, Wy: refleja un promedio ponderado de los valores de la 
variable en las regiones “vecinas” \cite{torres_ant}




