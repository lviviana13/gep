\section{Patrones Espaciales}

Aquie las ubicaciones pertenecen a un conjunto $D$ que puede ser
discreto o continuo y su selección no depende del investigador ($D$ aleatorio). Ejemplos de datos dentro de esta área son: Localización de nidos de pájaros en una región dada, puntos de imperfectos dentro de una placa metálica, ubicación de los sitios de terremoto en Colombia . Debe notarse
que en los ejemplos anteriores hay aleatoriedad en la selección de los sitios. En general el propósito de análisis
en estos casos es el de determinar si la distribución de los individuos dentro de la región es
aleatoria, agregada o uniforme.