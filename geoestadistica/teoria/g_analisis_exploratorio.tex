\subsubsection{Análisis exploratorio}

La parte exploratorio se basa en técnicas estadísticas convencionales que permiten obtener todo un conjunto de información, desconocida a priori sobre la muestra bajo estudio, que es imprescindible para realizar "correctamente" cualquier análisis estadístico y en particular un análisis geoestadístico.  \cite{notas_clase2}

Aquí se encuentra el cálculo de la  media, la mediana, obtención de histogramas, análisis de rangos intercuartílicos y de desviaciones estándar, el histograma, el diagrama de cajas, nubes direccionales, nube variográfica, entre otros. 

La nube variográfica nos habla de las diferencias cuadradas de las distancias de los pares de puntos que estamos realizando.

$$ \frac{[Z(s_i+h)-Z(s_i)]^2}{2}$$