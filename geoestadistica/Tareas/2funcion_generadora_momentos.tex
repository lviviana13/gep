\documentclass{article}
\usepackage[utf8]{inputenc}
\usepackage{amsmath}
\usepackage{amssymb}
\usepackage{natbib}
\usepackage{graphicx}


\title{Tareas Geoestadística}
\author{Laydi Viviana Bautista}
\date{March 2019}


\begin{document}


\paragraph{Ejercicio} Calcular la media con la función generadora de momentos para el lanzamiento de dados. 

Si lanzó un dado los posibles valores que puede tomar la variables son los siguientes:

\begin{tabular}[c]{|c|c|c|c|c|c|c|}
    \hline
     x & 1 & 2 & 3 & 4 & 5 & 6  \\ \hline
     y & $\frac{1}{6}$&$\frac{1}{6}$&$\frac{1}{6}$&$\frac{1}{6}$&$\frac{1}{6}$ & $\frac{1}{6}$ \\
     \hline
\end{tabular}

ahora, por definicíón la función generadora de momentos para una variable aleatoria es

$$m_x(t)=E[exp(tX)]=\underset{x}{\sum}exp(tX)p(x)$$ entonces la función generadora se puede expresar de la siguiente manera

\begin{equation*}
    m_x(t)  = e^{t1}\left( \frac{1}{6}\right) + e^{t2}\left( \frac{1}{6}\right) + e^{t3}\left( \frac{1}{6}\right)+e^{t4}\left( \frac{1}{6}\right)+e^{t5}\left( \frac{1}{6}\right)+ e^{t6}\left( \frac{1}{6}\right)
\end{equation*}


\begin{equation}
m_x(t)  = \left( \frac{1}{6} \right) \left( e^{t}+e^2t}+e^{3t}+e^{4t}+e^{5t}+e^{6t}\right) 
\label{fgm}
\end{equation}

Si derivamos la función generadora de momentos \ref{fgm} y evaluamos en t=0, obtenemos la media. 

\begin{equation*}
    \frac{dm_x(t)}{dt}|_{t=0} = \frac{d}{dt} \left(
    \left( \frac{1}{6} \right) \left( e^{t}+e^{2t}+e^{3t}+e^{4t}+e^{5t}+e^{6t}\right)
    \right)
\end{equation*}
\begin{equation*}
    \frac{dm_x(t)}{dt}|_{t=0} =
    \left( \frac{1}{6} \right) \left( 1+2e^{2t}+3e^{3t}+4e^{4t}+5e^{5t}+6e^{6t}\right)
\end{equation*}
\begin{equation*}
    \frac{dm_x(t)}{dt}|_{t=0} =
    \left( \frac{1}{6} \right) \left(1+2+3+4+5+6\right)
\end{equation*}
\begin{equation}
     \frac{dm_x(t)}{dt}|_{t=0} = \frac{7}{2} 
\end{equation}

\end{document}