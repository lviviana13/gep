


\section{Estadística Espacial}

La Estadística espacial es la reunión de un conjunto de metodologías apropiadas para el
análisis de datos que corresponden a la medición de variables aleatorias en diversos sitios, ya sean estps puntos del espacio o agregaciones espaciales de una región. 

De manera más formal se puede decir que la estadística espacial trata con el análisis de 
realizaciones de un proceso estocástico   $${ Z ( s ) : s \in D }$$  donde $S$ representa un ubicación en el espacio euclidiano $R^d$ d-dimensional y varía sobre un conjunto de índices $D \subset R^d$.  $Z(s)$ es una variable aleatoria en
la ubicación $s$ y el conjunto índice $D$ puede ser continuo, discreto o aleatorio.\cite{giraldo}


Esta ramificación de la Estadística surge, entre otras cosas, de la necesidad de generar métodos de análisi que se ajusten a las características de los datos espaciales, los cuales no cumplen, en particular, con el supuesto de independencia propuesto en el análisis clásico. La correlación entre los datos espaciales es "proporcional" a la distancia que separa a las ubicaciones en donde se realizaron las mediciones; esta premisa se puede soportar en una de las leyes de Tobler que dice "Todo esta relacionado con todo lo demás, pero cosas cercanas están más relacionadas que cosas distantes"\cite{notas_clase2}


La estadística espacial se subdivide en tres grandes áreas: La geoestadistica, lattices y patrones espaciales. La pertinencia de cada una
de ellas está asociada a las características del conjunto $D$ de índices del proceso estocástico
de interés. 

  
