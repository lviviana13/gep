\section{Lattices}

Las ubicaciones $s$ pertenecen a un conjunto $D$ discreto y son seleccionadas por el investigador ($D$ fijo). Estas pueden estar regular o irregularmente
espaciadas. Algunos ejemplos de datos en lattices son los siguientes: Tasa de morbilidad de hepatitis en Colombia medida por departamentos, tasa de accidentalidad en sitios de una ciudad, colores de los pixeles en interpretación de imágenes de satélite. En los ejemplos
anteriores se observa que el conjunto de ubicaciones de interés es discreto y que estas corresponden a agregaciones espaciales más que a un conjunto de puntos del espacio. Es obvio que la interpolación espacial puede ser carente de sentido con este tipo de datos. \cite{giraldo}