
\section{Geoestadística}


Las ubicaciones $s$ provienen de un conjunto $D$ continuo  y son seleccionadas
a juicio del investigador ($D$ fijo), es decir que el investigador puede hacer selección de puntos del espacio a conveniencia o seleccionar los sitios bajo algún esquema de muestro probabilístico. Algunos ejemplos  son: Niveles de un contaminante en diferentes sitios de una
parcela, contenidos auríferos de una mina, valores de precipitación en Colombia medida en
las diferentes estaciones meteorológicas en un mes dado o los niveles piezométricos de un
acuífero. En los ejemplos anteriores es claro que hay continuidad espacial, puesto que en
cualquier sitio de la parcela, de la mina, de Colombia o del acuífero pueden ser medias las
correspondientes variables.

Es importante resaltar que en geoestadística el propósito esencial es la interpolación y si no hay continuidad espacial pueden hacerse predicciones carentes de sentido. \cite{giraldo}

Cuando el objetivo es hacer predicción, se realiza dos etapas: análisis estructural (describir la correlación en puntos del espacio) y la segunda etapa es la predicción en sitios no muestrarios por medio de Kriging (proceso que calcula el promedio ponderado de las observaciones muestrales). Los pesos asignados a los valores muestrales son apropiadamente determinados por la estructura espacial de correlación establecida en la primera etapa y por la configuración de muestro (Petitgas, 1996 citado por \cite{giraldo})

La primera etapa en el desarrollo de un análisis geoestadístico es la determinación de la dependencia espacial entre los datos medidos de una variable. Esta fase es también conocida como análisis estructural. \cite{giraldo}

