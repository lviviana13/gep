\section{Dependencia Espacial}

La dependencia espacial hace referencia a la estructura de correlación de las variables aleatorias del proceso
${Z(s) : s \in D \subset R^d }$. Cuando hay dependencia espacial los sitios
cercanos tienen valores más similares que los distantes. Por el contrario la ausencia de correlación espacial se refleja en el hecho de que la distancia entre los sitios no tiene influencia
en la relación de sus valores. \cite{giraldo}

\subsection{Test de Moran}

Este test es especialmente usado en datos de áreas. Sean $Z(s_1 ), · · · , Z(s_n)$, las variables medidas en las $n$ áreas. La noción de autocorrelación espacial de estas variables está asociada con la idea de que valores observados en áreas geográficas adyacentes serán más similares que los esperados bajo el supuesto de independencia espacial. El ı́ndice de autocorrelación de Moran considerando la información de los vecinos más cercanos es definida
como:

$$ I = \frac{n}{\Sigma}$$

Valores positivos (entro 0 y 1) indican autocorrelación directa (similitud entre valores cercanos) y valores negativos (entre -1 y 0) indican autocorrelación inversa (disimilitud entre las áreas cercanas). Valores del coeficiente cercanos a cero apoyan la hipótesis de aleatoriedad espacial.

Para el cálculo del índice de Moran es necesario definir la proximidad entre las áreas. Lo anterior se lleva a cabo por medio del cálculo de una matriz de proximidad espacial. 







