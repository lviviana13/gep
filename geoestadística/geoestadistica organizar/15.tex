\documentclass{article}
\usepackage[utf8]{inputenc}
\usepackage{amsmath}
\usepackage{amssymb}
\usepackage{natbib}
\usepackage{graphicx}

\title{Ajuste Semivariograma}
\author{Laydi Viviana Bautista}
\date{March 2019}


\begin{document}

%\maketitle

\section{Marco Teorico}

La geostadística es una manera de describir la continuidad espacial de cualquier fenómeno natural. Con ello llegamos a conocer la forma en que varía cualquier variable continua en el espacio (patrón espacial) a una o varias escalas seleccionadas, con un nivel de detalle que permite cuantificar la variación espacial de la variable en distintas direcciones del espacio.

La geostadística utiliza funciones para modelar esta variación espacial, y estas funciones son utilizadas posteriormente para interpolar en el espacio el valor de la variable en sitios no muestreados. 

La fortaleza de la geostadística es que esta interpolación (conocida como kriging) es considerada una estima muy robusta ya que se basa en la función continua que explica el comportamiento de la variable en las distintas direcciones del espacio, y que en contraste con otros métodos de interpolación (como por ejemplo interpolar un punto usando los valores de los puntos que le rodean ponderados por la distancia que los separa) permite asociar la variabilidad de la estima (conocido como grado de incertidumbre).

La geostadística permite por tanto responder a las siguientes preguntas: ¿Cuál es el patrón espacial de mis variables de interés? ¿A qué escala se repite este patrón espacial?¿Existe covariación espacial entre las distintas variables de interés?¿Cuál es la
mejor representación gráfica de la continuidad de mi variable?¿Cual es el grado de incertidumbre de estas estimas? Las
respuestas a estas preguntas son siempre dependientes de la escala espacial elegida. Por ejemplo, si queremos relacionar la
distribución de especies de plantas con el contenido en humedad del suelo, los resultados serán distintos si muestreamos en
un cuadrado de 5 x 5 m que si lo hacemos en un cuadrado de 100 x 100 m. Nos encontraremos dos diferentes patrones
espaciales (aunque el mayor engloba, y a veces oculta al menor) y dos diferentes grados de incertidumbre que podremos
asociar a la respuesta a nivel de individuo a la variación en humedad (parcela pequeña) o a la respuesta de poblaciones o
comunidades a dicha variación (parcela grande)

\subsection{}

\subsection{El semivariograma empírico } La función básica que describe la variabilidad espacial de un fenómeno de interés se conoce como semivariograma. El
semivariograma responde a la siguiente pregunta ¿Cómo de parecidos son los puntos en el espacio a medida que estos se
encuentran más alejados? Imaginemos la parcela de la Figura 1.


\bibliography{sample.bib}

\end{document}