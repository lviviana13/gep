\section{Estadística Espacial}

La Estadística espacial es la reunión de un conjunto de metodologías apropiadas para el
análisis de datos que corresponden a la medición de variables aleatorias en diversos sitios
(puntos del espacio o agregaciones espaciales) de una región. 

De manera más formal se puede decir que la estadística espacial trata con el análisis de 
realizaciones de un proceso estocástico ${ Z ( s ) : s \in D }$ , en el que $s \in R^d$ representa 
una ubicación en el espacio euclidiano d-dimensional, $Z(s)$ es una variable aleatoria en
la ubicación $s$ y $s$ varía sobre un conjunto de índices $D \subset R d$. \cite{giraldo}

La estadística espacial se subdivide en tres grandes áreas. La pertinencia de cada una
de ellas está asociada a las características del conjunto $D$ de índices del proceso estocástico
de interés. \cite{giraldo}

\subsection{Geoestadística}

Las ubicaciones $s$ provienen de un conjunto $D$ continuo  y son seleccionadas
a juicio del investigador ($D$ fijo). Es importante resaltar que en geoestadísica el proposósito esencial es la interpolación y si no hay continuidad espacial pueden hacerse predicciones carentes de sentido.  

\subsection{Lattices} 
