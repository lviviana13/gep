\documentclass{article}
\usepackage[utf8]{inputenc}
\usepackage{amsmath}
\usepackage{amssymb}
\usepackage{natbib}
\usepackage{graphicx}


\title{Tareas Geoestadística}
\author{Laydi Viviana Bautista}
\date{March 2019}


\begin{document}

\section{Datos de área}

Podemos tener dos enfoques dependiendo las caracterisiticas de nuestros datos

Entonces, si tenemos un $Z_i$ variable regionalizada, que se encuentra en un dominio

- Matrices o enmallados regulares
- Enmallados irregulares \footnote{Estos son los que más se encuentran en la práctica}


\subsection{Principios cuando trabajamos en el tratamiento de datos espaciales}

\paragraph{Interdependencia} Todo lo que sucede en el espacio está "dependiendo" (Zonas vecinas). Aquí hay que tener en cuenta que las coneiones pueden ser físicas (vías) o simbolicas (tratados)

\paragraph{Asimetria} Es decir, los movimientos en un dirección no se presenta en la otra dirección (Ejemplo: Si un país importa un producto no lo va a exportar)

\paragraph{Utopia}debemos buscar la causa de un problema de una región en otras regiones

\paragraph{No linealidad} 

\paragraph{Inclusión de variables topologicas} Distancia, coordenadas




\end{document}